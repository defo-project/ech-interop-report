\section{Software}

This section describes libraries, clients and servers that support ECH.

\subsection{Libraries}

The libraries listed in Table \ref{tab:echlibs} have some support for ECH.  Of
the libraries listed, ECH support for all of the OpenSSL variants, Python,
libcurl and Conscrypt, were developed by the DEfO project.

\tiny
\begin{longtblr} [
        caption = {Libraries with ECH},
        label = {tab:echlibs}
    ] {
        colspec = {| l | p{0.7\linewidth} |},
        rowhead = 1
    }
    \hline
        Library Name & Details\\

    \hline
        OpenSSL-feature-branch & Source: \url{https://github.com/openssl/openssl/tree/feature/ech}\\
        & Version: 3.6.0-dev feature branch in upstream repo\\
        & ECH support: ECH shared-mode client and server\\
        & very basic ``s\_client'', TLS only (no DTLS, no QUIC)\\
        & the OpenSSL project's target for ECH PRs\\
        & ECH code will end up here before being merged to master\\

    \hline
        OpenSSL-defo & Source: \url{https://github.com/defo-project/openssl/}\\
        & Version: 4.0.0-dev fork of master, same APIs as OpenSSL-feature-branch, but full ECH code\\
        & ECH support: ECH shared and split mode client and server, TLS only (no DTLS, no QUIC)\\
        & this is used for DEfO project CI builds and tests, and is rebased to upstream frequently\\

    \hline
        OpenSSL-sftcd & Source: \url{https://github.com/sftcd/openssl/tree/ECH-draft-13c}\\
        & Version: 3.4.0-dev fork of master\\
        & ECH support: client and server, TLS only (no DTLS, no QUIC)\\
        & this was the DEfO project's original main development branch, but is now obsoleted\\

    \hline
        boringssl & Source: \url{https://boringssl.googlesource.com/boringssl}\\
        & Version: boringssl doesn't do versions, last local build 2025-10-12\\
        & ECH support: ECH shared-mode client and server, ECH for TLS, QUIC and (possibly) DTLS\\
        & in production use (Chromium, et al), a little more limited in HPKE suites than OpenSSL - only KEM is x25519 \\

    \hline
        AWS-LC & Source: \url{https://github.com/aws/aws-lc}\\
        & Version: 1.62.0\\
        & ECH support: ECH client and server, ECH for TLS (unsure of DTLS/QUIC)\\
        & A versioned library based on boringssl \\
        & Seems to work as well in terms of ECH interoperability\\

    \hline
        NSS & Source: \url{https://github.com/nss-dev/nss.git}\\
        & Version: NSS 3.116\\
        & ECH support: ECH client and server, ECH for TLS (unsure of DTLS/QUIC)\\
        & in production use (Firefox), a little more limited in HPKE suites than OpenSSL - only KEM is x25519\\

    \hline
        WolfSSL & Source: \url{https://github.com/wolfSSL/wolfssl.git}\\
        & Version: 5.8.2\\
        & ECH support: ECH client and server, ECH for TLS (unsure of DTLS/QUIC)\\
        & - ECH not built by default (needs ``--enable-ech'')\\
        & - fails when HelloRetryRequest seen - \url{https://github.com/wolfSSL/wolfssl/issues/6802}\\

    \hline
        gnuTLS & Source: \url{https://gitlab.com/gnutls/gnutls/blob/master/README.md}\\
        & Version: work-in-progress\\
        & ECH support: interop untested (by DEfO-project) at this time\\
        & an ECH merge request exists but has yet to be merged \url{https://gitlab.com/gnutls/gnutls/-/merge_requests/1748}\\
        & Seemingly no progress since Version 1 of this report\\

    \hline
        golang & Source: \url{https://go.googlesource.com/go} or \url{https://github.com/golang/go/}\\
        & Version: 1.23 or later required for ECH client, current version: 1.24.4\\
        & ECH support: ECH shared-mode client and server\\

    \hline
        rustls & Source: \url{https://github.com/rustls/rustls/}\\
        & Version: 0.23.32 \\
        & ECH support: client only\\

    \hline
        CPython & Source: \url{https://github.com/defo-project/cpython}\\
        & Version: CPython 3.13/3.14 \\
        & ECH support: TLS client only \\
        & - CPython tests were just (2024-12-05) added to our smokeping tests\\

    \hline
        libcurl & Source: part of \url{https://github.com/curl/curl}\\
        & Version:  8.17.0-DEV \\
        & ECH support: TLS client only \\

    \hline
        Conscrypt-GP & Source: https://github.com/guardianproject/conscrypt\\
        & Version: 2.6-alpha\\
        & ECH support: only ECH client is relevant\\
        & This is based on boringssl and only runs on Android\\

    \hline

\end{longtblr}
\normalsize

\subsection{Clients}

Current versions of Firefox (e.g. 143.0, used in our smokeping tests) support
ECH by default.
Firefox does, however, automatically disable ECH when it detects the presence
of ``family safety" software
\footnote{https://support.mozilla.org/en-US/kb/faq-encrypted-client-hello}.
We have received anecdotal evidence that this detection mechanism also
triggers when detecting ISP-level and state-imposed censorship, preventing
the use of ECH in those situations.

The Tor Browser (a Firefox derivative) doesn't support ECH by default, as it
is currently built from an older ``Extended Support Release" that only
enables ECH when DNS-over-HTTPS (DoH) is enabled and Tor Browser prevents
enabling DoH to prevent that being used as a user fingerprinting vector.
To test if a browser supports ECH one can visit \url{https://test.defo.ie/}
or, for much more detail \url{https://test.defo.ie/iframe_tests.html}.

Current versions of Chromium (version 140.0.7339.185 is used in our tests)
also support ECH by default.
Other Chromium-derived browsers also support ECH, while we don't include them
in our ongoing tests described below, we have manually verified that current
versions of Vivaldi, Brave and Opera also have ECH support.

Browsers in general are still somewhat quirky in terms of whether or not
ECH will succeed, even when ECH is properly setup. This seems at least
partly due to the delay that can inherently occur between the browser
receiving DNS answers for A/AAAA records, and the time at which the
browser receives a DNS answer for an HTTPS record (containing ECH
information). It seems to be the case that browsers may be ``impatient''
in that they start the TLS session without attempting ECH if that
delay is too long, and that that happens sufficiently often to be
noticeable. (Note however, that this report does not in itself yet
justify that claim, but work is ongoing.)

As this ``impatience'' has seemingly disimproved since version 1 of
this report we investigated further to see if we could determine for
sure that our suspiscions were correct. We did this with a variant
of our smokeping script for chromium that use localhost port 53 for
it's DNS recursive. In that setup we could see the IMCP error messages
('port unreachable') when the recursive was attempting to provide HTTPS RR answers.
This seems to confirm that the browser has given up waiting for the HTTPS
RR answer. We also confirmed this via chromium ``netlog'' traces where
we see ``ERR\_DNS\_TIMED\_OUT''. Informal discussion with prople more
familiar with chromium internals indicates that this situation may be
a result of balancing contradictory requirements - some scenarios 
apparently call for failing quickly while others (such as ours) suffer
from that impatience. As there does not appear to be a way to control
those chromium-internal DNS timeouts in our tests, we properly report
these unexpcted test results as failure cases.

The only command line tool supporting ECH of which we're aware is curl. The
DEfO project
contributed ECH code as an experimental feature for curl that was merged to the
master branch in April 2024. As an experimental feature, ECH is not built by
default nor part of a release, so is only currently available to those who
build from source. Curl supports using OpenSSL, boringssl or Wolfssl libraries
for ECH.  Build instructions are at
\url{https://github.com/curl/curl/blob/master/docs/ECH.md}.

As a scripting tool, often used as part of a TLS client, Python also fits here.
The DEfO project CPython build adds ECH support to Python via the OpenSSL
library.  That build is available at
\url{https://github.com/defo-project/cpython} with the new APIs required for
ECH described at
\url{https://irl.github.io/cpython/library/ssl.html#encrypted-client-hello}.

Conscrypt is Android's library for TLS. It is used in three forms: built into
Android OS, a system-wide standalone TLS provider, and a library to provide TLS
in an app. Our fork of that has ECH support via boringssl from Google 
which implements the Android side. Since it is quite unlikely that a
multi-domain web service would ever run on Android, only the ECH client side is
relevant.

\subsection{Servers}

\todo{Got here, update away}

The servers listed in Table \ref{tab:servers} support ECH. In each case,
the ECH integration, using OpenSSL, was developed as part of the DEfO
project.

\tiny
\begin{longtblr} [
        caption = {Servers supporting ECH},
        label = {tab:servers}
    ] {
        colspec = {| l | p{0.7\linewidth} |},
        rowhead = 1
    }
    \hline
        Server Name & Details\\

    \hline
        lighttpd-upstream & Source: \url{https://github.com/lighttpd/lighttpd1.4.git}\\
        & Version: 1.4.78\\
        & Comment: Upstream supports shared-mode ECH if built with ECH-enabled boringssl or OpenSSL\\

    \hline
        lighttpd-defo & Source: \url{https://github.com/defo-project/lighttpd}\\
        & Version: 1.4.78\\
        & Comment: as for lighttpd-upstream but to enable these tests is built with\\
        & ``CFLAGS=-DLIGHTTPD\_OPENSSL\_ECH\_DEBUG'' \\

    \hline
        nginx & Source: \url{https://github.com/defo-project/nginx}\\
        & Version: 1.27.3.4\\
        & Comment: both shared-mode and split-mode ECH\\

    \hline
        apache2 & Source: \url{https://github.com/defo-project/apache-httpd}\\
        & Version: 2.5.0.ech.351\\
        & Comment: only shared-mode ECH\\

    \hline
        haproxy & Source: \url{https://github.com/defo-project/haproxy}\\
        & Version: 3.2.dev0.62\\
        & Comment: both shared-mode and split-mode ECH\\

    \hline
        OpenSSL s\_server & Source: \url{https://github.com/defo-project/openssl}\\
        & Version: 3.5.0-dev\\
        & Comment: shared-mode only but can easily ``force'' HelloRetryRequest (HRR)\\
        
    \hline

\end{longtblr}
\normalsize
