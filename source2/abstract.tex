\begin{abstract}
    Encrypted Client Hello (ECH) is a privacy-enhancement for the Transport Layer
    Security (TLS) protocol that underlies all web security and the security of
    many other Internet protocols. While the specification for ECH is relatively mature,
    (though not yet an Internet-RFC), and while implementations are already widespread,
    work remains to ensure that a random client and server can successfully use ECH.
    To that end, this report describes software libraries, client and server software
    packages, and Internet services for which ECH is relevant, and configurations of
    those where ECH currently works, or fails, as part of overall efforts to encourage
    ECH deployment and hence improve Internet security and privacy. At the time of
    writing, the overall interoperability story for ECH could be summed up as:
    if you stick to a simple deployment that works with browsers, that'll be fine,
    but if you want to push out the boundaries, then something will break, so don't
    do that. We justify that claim via a test setup involving 67 ECH test URLs 
    (some with broken or non-normal setups), six test clients, and coverage of
    almost all known ECH-capable server technologies. Currently, our
    tests indicate that things are more than 90\% ``OK'' but are not yet more than
    95\% ``OK''.
\end{abstract}
