\section{Conclusions}

Based on all the above, we can reach some conclusions:

\begin{itemize}

    \item
        At the ECH protocol level, interoperability, ignoring broad ECH
        ciphersuite support, is very good. The (real) issues in getting
        interoperable wide deployment of ECH revolve around complexity in HTTPS
        RR processing.

    \item
        If deploying ECH: Keep it simple, because ECH ciphersuite handling is
        not at all flexible in many clients. Only publish a ``singleton''
        ECHConfigList in one HTTPS RR and use x25519/HKDF-SHA256/AES128 for
        that one ECHConfig. Do not use aliasMode. If rotating ECH keys, do keep
        ``old'' private values usable at servers for a duration that is a
        multiple (e.g. 3x or 5x) of the duration for which the ``current'' ECH
        public value is published in the DNS. It's probably wise to also have a
        shorter TTL on HTTPS RR values, perhaps half of the key rotation
        duration or less.

    \item It is very non-trivial for non-browser clients to fully handle HTTPS
        RRs.  For typical non-browser clients HTTPS RR processing requires a
        change from the very simple situation of using the system resolver for
        A/AAAA RRs to a new situation where far more complex HTTPS processing
        may be needed. We expect a possibly long transition period where such
        clients have very simplistic processing of HTTPS RRs for ECH, likely
        just processing a single HTTPS RR value. Again: deployers will need to
        keep it simple, and possibly for an extended period.

    \item
        Browsers are complex beasts. It is very likely that real browser
        behaviour will be better than indicated in our test results above,
        which all involve accessing one single URL with fully ``cold'' caches.
        However, our ECH trial \cite{echtrial} indicates that one can't be
        overly confident that ECH will always succeed even if the browser
        user-agent string indicates use of a browser-version that is capable of
        ECH success.  So, one cannot be 100\% confident that ECH will always be
        used, and always succeed, for browser users, even with the most
        conservative deployment. However, one can likely be confident that in
        such scenarios, ECH will often be used and will often succeed.

\end{itemize}

As a last word, the author is happy to use the benefit of hindsight and
speculate that had a far simpler form of ECH been developed as a first version,
with no options at all, and had that design been deployed and iterated upon, we
would be much closer to seeing ECH widely deployed today with excellent
interoperability.  The lesson there however, is for protocol designers, and not
for those implementing and deploying the current ECH specification, who are the
main target for this document.
