\section{Existing Implementations}

Several open-source QUIC implementations exist. This report considers the Open Source implementations that are actively evaluated by the QUIC interoperability runner~\cite{interop}, which performs automated tests to ensure the implementations are compatible with each other. Differences may arise where the standard has not given enough detail on a specific aspect of the protocol design, and this testing is expected to highlight differences in interpretation.

ECH support was determined from analysing the source code for these implementations in Table~\ref{tbl:quics-ech}.

\begin{table}[ht]
\centering
\begin{tabular}{l|l|l|l}
QUIC Impl. &  QUIC Developer & TLS Library  & ECH Support  \\
\hline \hline
\color{red}{\texttt{mvfst}} & Facebook &  fizz  & \color{red}{Yes} \\
\hline
\color{red}{\texttt{neqo}} & Mozilla & NSS  & \color{red}{Yes} \\
\hline
\color{red}{\texttt{chrome}} & Google & BoringSSL  & \color{red}{Yes} \\
\hline
\texttt{quiche}  & Cloudflare & BoringSSL & No \\
\hline
\texttt{msquic}  & Microsoft  & OpenSSL & No \\
\hline
\texttt{xquic}  & Alibaba & BoringSSL, BabaSSL & No \\
\hline
\texttt{s2n-quic}  & AWS  & rustls, s2n-tls & No \\
\hline
\texttt{lsquic} & LiteSpeed  & BoringSSL &  No \\
\hline
\color{red}{\texttt{ngtcp2}} & T. Tsujikawa  & BoringSSL &  \color{red}{Yes}  \\
\hline
\texttt{picoquic} & C. Huitema  & OpenSSL & No  \\
\hline
\texttt{quic-go} & M. Seemann & Go Std. Lib.& No  \\
\hline
\texttt{kwik} & P. Doornbosch & Agent15 & No \\
\hline
\texttt{aioquic} & J. Lainé   & OpenSSL & No  \\
\hline
\texttt{quinn}  & D. Ochtman, B.Saunders & rustls  & No  \\

\end{tabular}
\caption{ECH Support across QUIC implementations}
\label{tbl:quics-ech}
\end{table}

Before starting the handshake, QUIC also has to provide TLS with the transport parameters~\footnote{\url{https://www.iana.org/assignments/quic/quic.xhtml}} that it wishes to carry. This is an additional requirement from TCP~\cite{rfc9000}. As per RFC 9001, the \texttt{quic\_transport\_parameters} extension has to be carried in a TLS ClientHello and the EncryptedExtensions messages during the handshake.
Differences regarding transport parameter handling exist within the four QUIC implementations that were found to support ECH, with evidence that \texttt{neqo} allows different transport parameters in the inner and outer ClientHello, however cautioning against using this feature unless the ECH configuration has been validated, due to a mechanism that filters sensitive transport parameters in the outer ClientHello (see \href{https://github.com/mozilla/neqo/blob/main/neqo-crypto/src/agent.rs#L1176-L1182}{source code}).
For the other implementations (\texttt{mvfst}, \texttt{chrome}, and \texttt{ngtcp2}), this behaviour appears to depend on the underlying TLS library, however no mechanism for filtering sensitive transport parameters was noted in the libraries BoringSSL and fizz.

%I think BoringSSL allows the transport parameter extension in the Outer ClientHello (see https://github.com/google/boringssl/blob/main/ssl/extensions.cc#L3640-L3646 and https://github.com/google/boringssl/blob/main/ssl/encrypted_client_hello.cc#L875-L880), but fizz does not.

Some content carried in transport parameters could potentially benefit from being protected by ECH encryption. Determining which parameters, and under what circumstances, should not be carried in an Outer ClientHello will require further analysis best undertaken during implementation.
For example, \texttt{neqo} retains only the parameters necessary to ensure the connection attempt succeeds, as well as any that might affect connection configuration in ways that could impact operation\footnote{These include \texttt{original\_destination\_connection\_id}, \texttt{stateless\_reset\_token}, \texttt{initial\_source\_connection\_id}, \texttt{retry\_source\_connection\_id} and \texttt{version\_information}. Other parameters are included (\texttt{ack\_delay\_exponent}, \texttt{max\_ack\_delay}, and \texttt{max\_udp\_payload\_size}) on the basis they pose no privacy risk (see \href{https://github.com/mozilla/neqo/blob/d46f3f4a9454bb50fc40f6251b43d43442a5db00/neqo-transport/src/tparams.rs\#L749-L782}{source code})}.
