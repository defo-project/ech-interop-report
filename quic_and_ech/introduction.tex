\section{Introduction}

Deployments of the TLS~\cite{rfc8446} protocol
expose the name of the server (e.g. the web site DNS name) via the Server Name
Indication (SNI) field in the first message sent (the ClientHello).  The
Encrypted Client Hello (ECH)~\cite{draft-ietf-tls-esni} extension to TLS is a
privacy-enhancing scheme that aims to address this leak.

This report discusses considerations for deploying ECH with the QUIC transport protocol, assuming the QUIC implementation relies on an open-source TLS library, such as OpenSSL.

The primary audience for this document are those implementing and
deploying ECH, as well as those implementing QUIC. Secondarily, there may be lessons to learn for those
designing protocols like ECH.

The Open Technology Fund (OTF, \url{https://www.opentech.fund/}) have
funded the DEfO project (\url{https://defo.ie}) to develop
ECH implementations for OpenSSL, and to otherwise encourage implementation
and deployment of ECH.
As we expect the implementation and deployment environment for ECH to change
over time, this report will be updated as events warrant and is currently
versioned based on the build-time of this PDF.

